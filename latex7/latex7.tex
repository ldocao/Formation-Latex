\documentclass[a4paper]{article}

\usepackage{hyperref}
\usepackage[utf8]{inputenc}
\usepackage[francais]{babel}
\usepackage[margin=1.5in]{geometry}

\usepackage{amsmath}
\usepackage{amssymb}





\begin{document}

\tableofcontents



\section{Introduction}\label{section1}



Lorem ipsum dolor sit amet, consectetur adipiscing elit. Donec et pretium felis. Mauris eu dictum neque. Proin id ultricies sapien, ac accumsan enim. Praesent vulputate enim orci, commodo cursus dolor congue sodales. Etiam tristique mauris in purus iaculis lobortis. Aliquam id elit velit. Vestibulum sit amet lobortis leo. Donec nisl sem, vestibulum non orci eu, finibus vehicula lacus. Fusce et lobortis purus, nec malesuada nisi. Nunc lobortis quam id fringilla auctor. Phasellus ac quam nec orci lacinia interdum id sed est. Sed blandit ultrices purus, sed posuere dolor cursus at. In quis dui nec tellus ultrices tincidunt.

\begin{equation}
  \label{eq:73}
  \sigma_{xx} = \sigma_{yy} = \sigma_{zz} = -P
\end{equation}

Etiam at urna erat. Fusce ac ex augue. Proin finibus massa nec lectus accumsan tristique. Vestibulum iaculis accumsan rhoncus. Nulla nisi leo, tempor vitae eros ac, gravida auctor eros. Maecenas convallis pulvinar mauris quis tristique. Aliquam lacinia ullamcorper blandit.

Je fais référence à la section \ref{section1}.


\section{Conclusion}
Duis ut nisi ut magna dignissim porttitor. Integer sed velit id nibh sagittis laoreet ac ut tellus. Maecenas suscipit dignissim nulla, id pulvinar tortor aliquam at. Cras ut risus est. Quisque et elementum arcu. In porta tellus metus, in ultrices quam elementum ut. Proin vitae consectetur sapien. Fusce vitae nisi diam. Mauris at eros erat. Mauris eget tempus sapien. Vestibulum tempor lacinia lobortis.

\begin{equation}
  \label{eq:39}
  \sigma_{ij} = -P\delta_{ij} +D_{ij} 
\end{equation}


La formule de Pythagore:

\begin{equation}
a^2=b^2+c^2 \label{pythagore}
\end{equation}

Maecenas sit amet velit tristique, pellentesque neque a, tincidunt leo. Mauris at suscipit turpis. Suspendisse vitae faucibus justo, non euismod justo. Nulla molestie volutpat nunc, a placerat felis varius sit amet. Mauris a interdum neque. Aenean commodo augue a tellus semper, sit amet accumsan odio sagittis. Vivamus eleifend neque in purus condimentum congue. Duis dictum sapien diam, et mollis urna maximus a. Nullam lacinia orci sit amet dui commodo, sit amet tempor eros volutpat. Nunc auctor sagittis mauris, sollicitudin molestie justo finibus ac. Aliquam erat volutpat. Sed pellentesque ultrices orci, in tristique augue. Mauris vehicula est tortor. Nulla ut sollicitudin erat, quis vestibulum lacus. Vestibulum ex massa, tristique tempor sapien eu, sollicitudin ultrices eros. In porta felis orci, eget dignissim nisl dictum vel.

\begin{equation}
  \label{eq:101}
  A_\phi=\frac{\sin\theta}{r^2}
\end{equation}

Curabitur odio nisl, aliquam sit amet faucibus eu, viverra eu libero. Vivamus ac ligula nec odio posuere egestas sed eget arcu. Quisque molestie, arcu at sodales sodales, justo diam vehicula sapien, nec faucibus nulla dolor nec sem. Maecenas purus lacus, pretium ac purus et, convallis euismod eros. Nullam nec quam nulla. Aliquam elementum consequat odio sed gravida. Integer dignissim bibendum orci quis efficitur.

Pour calculer l'hypothénuse d'un triangle, il suffit d'appliquer l'équation \ref{pythagore}.

Bonjour, bienvenu sur screencastmint, je vais vous poser quelques questions:

\begin{enumerate}
\item Quel est ton nom ?
\item Quel est ton âge ? \label{age}
\item Quel âge auras tu dans 23 ans ? Aidez vous de la question \ref{age}.
\end{enumerate}


\end{document}