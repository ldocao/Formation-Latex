\documentclass[a4paper]{report}

%%PACKAGES
\usepackage[margin=2in]{geometry}
\usepackage[utf8]{inputenc}
\usepackage[francais]{babel}

\usepackage{amssymb} %symboles additionnels
\usepackage{amsmath} %ameliorations rendus des formules mathematiques



\begin{document}

Formation \LaTeX, 4ème vidéo

Ceci est ma première équation $a=1$ avec la méthode dollars.

Ceci est ma première équation a=1 avec la méthode dollars.

Une lettre grecque $\alpha=1$ au milieu du texte.

Je vais écrire ma deuxème équation

\begin{equation}
c=3
\end{equation}

%%à ne pas faire
%\begin{center}
%c=3
%\end{center}


\begin{equation}
x_{0}^{2n}
\end{equation}

Vous pouvez omettre les symboles accolades

\begin{equation}
x_0^{2n}
\end{equation}


Cela ne marche pas avec plusieurs caractères dans l'accolade

\begin{equation}
x_0^2n
\end{equation}

On peut introduire des espaces insécables $L=3~ \rm cm$. Le nombre d'espaces importe peu $\gamma =          2 Hz$.

Il existe beaucoup d'autres commandes similaires pour afficher en roman

\begin{equation}
\cos\alpha +\sin\beta = \exp a + \log b
\end{equation}

Une dernière commande utile :

\begin{equation}
\frac{a}{b}\cos\alpha=\frac{2\frac{4}{5}}{3}
\end{equation}


\end{document}
